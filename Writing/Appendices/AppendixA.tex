\chapter{Aggregation Functions}
\subsection {Decomposable functions}
For some aggregation function, one node may need to perform a single computation operation involving all the elements of the multiset, requiring more resources than the ideal ones. So, in order the distributed the effort to compute the multiset, there are some aggregation function that are decomposable. Meaning that, the effort could be done in a distributed way. A definition for decomposable function is also given in \cite{journals/corr/abs-1110-0725}:
\begin{definition} An aggregation function $ f : \mathbb{N}^I \to O$ is said to be self decomposable if, for some (merge) operator $\diamond$ and all non empty multisets $X$ and $Y$:
\begin{equation*}f(X \uplus Y) = f(X) \diamond f(Y) \end{equation*}
\end{definition}
The $\uplus$ denotates the standard multiset sum. The operator $\diamond$ is commutative and associative \cite{journals/corr/abs-1110-0725}. Some functions that are self-decomposable:
\begin{align*}SUM ({x}) &= x,\\
SUM(X \uplus Y) &= SUM(X)+SUM(Y).\end{align*}
\begin{align*}COUNT ({x}) &= x,  \\
COUNT(X \uplus Y)& = COUNT(X)+COUNT(Y).\end{align*}
\begin{align*}MIN ({x}) &= x,\\
MIN(X \uplus Y) &= MIN(X) \sqcap MIN(Y).\end{align*}
\begin{definition}
An aggregation function $ f : \mathbb{N}^I \to O$ is said said to be decomposable if for some function $g$ and a self-decomposable aggregation function $h$, it can be expressed as:
\begin{equation*}f=g \circ h\end{equation*}
\end{definition}
As the definition above, stated in \cite{journals/corr/abs-1110-0725}, self decomposable functions are a subset of the decomposable functions. One example of a decomposable functions $AVERAGE$:
\begin{align*} 
AVERAGE(X) &= g(h(X)),\\
h({x}) &= (x,1),\\
h(X \uplus Y) &= h(X) + h(Y),\\
g((s,c)) &= s/c. \end{align*}
Another example is the $RANGE$ which gives the difference between the maximum and minimum value.


\subsection {Duplicate sensitiveness and idempotence} 
For some functions, the presence of duplicate results does not affect the result. Examples of this aggregation functions are $MAX$ and $MIN$, where"\textit{ the result on only depend on its \textit{support} set(obtained by removing all duplicates)}"\cite{journals/corr/abs-1110-0725}. Others, like $SUM$ or $COUNT$, the duplicate numbers are relevant. This propriety is called duplicate sensitiveness, it is relevante in distributed aggregation. Using an idempotent binary operator on the elements of the multiset helps obtaining fault tolerance \cite{journals/corr/abs-1110-0725}.
\begin{definition}
An aggregation function $f$ is said to be duplicate insensitive if for all multiset $M, f(M) = f(S)$, where $S$ is the support set of $M$.
\end{definition}
A taxonomy table of aggregation is in \cite{journals/corr/abs-1110-0725} and it is presented below.
\begin{center}    
\begin{tabular}{|c||c|c|c|}
    \hline
                                         &   \multicolumn{2} {|  c  |}{ Decomposable}                                               &    Non-decomposable \\ \hline
                                         &    Self-Decomposable      &                               &  \\ \hline
      Duplicate insensitive  &    $MIN,MAX$                  &     $RANGE$         &  $DISTINCT,COUNT$ \\ \hline
      Duplicate sensitive     &    $SUM,COUNT$           &     $AVERAGE$     &  $MEDIAN,MODE$ \\ \hline
    
    \end{tabular}
\label{Taxonomy of aggregation functions}
\end{center}
