\chapter*{Abstract}
	The modern and globalized world become more and more dependent on energy sources such as electricity or oil. The power grids all over the planet become increasingly bigger leading to problems of energy waste and sustainability. The smart grids appear to address this issues and also to interconnect the various renewable energy sources like solar or wind since they have different requirements than the traditional ones. This new concept of grid introduce ICT and computation  and that's why they are called 'smart'. One of the biggest advantages of smart grids is the ability to remotely read fine-granular measurements from each smart meters, which enables the grid operators to balance load efficiently and offer adapted time dependent tariffs. With data aggregation from the smart meters it's possible to reduce the energy consumption and change it to be more efficient.
In this work we address the problem of smart metering data aggregation. We propose a distributed data aggregation approach, where all the smart meter sense the consumption data and some of them can work as aggregators as well. We also focus in observe how the aggregation algorithms work in the smart grid, collecting the results and evaluate which algorithm suites the best.

	\cleardoublepage

\chapter*{Resumo}
	O mundo moderno e globalizado tornou-se cada vez mais dependente em energias como a electrecidade ou o petroleo. As redes de energia por todo o planeta tornaram-se cada vez maiores levando a problemas de perda de energia e de sustentabilidade. As \textit{Smart Grids} apareceram para fazer face a esses problemas assim como interligar as varias fontes de energia renovaveis como a solar ou a eolica pois elas tem diferentes requisitos das tradicionais. Este novo conceito de rede introduz computacao e TIC e e por isso que elas sao denominadas de \textit{smart}. Uma das grande vantagens das \textit{Smart Grids} e a habilidade para ler dados de fina granularidade de maneira remota dos leitores inteligentes, permitindo aos operadores de rede balancear de forma eficiente a energia e oferecer tarifas adaptaveis. Com a agregacao dos dados dos  leitores inteligentes  possivel reduzir o consumo de energia e muda-lo para ser mais eficiente.\\
Neste trabalho sera tratado o problema da agregacao de dados de forma distribuida.  E proposta uma abordagem distribuida de agregacao, onde todos os leitores inteligentes leem o consumo de energia e alguns funcionam como agregadores.